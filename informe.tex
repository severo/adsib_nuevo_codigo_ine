\documentclass[letterpaper]{article}
\usepackage[spanish]{babel}
\usepackage[utf8]{inputenc}
\usepackage{mathtools}
\begin{document}
	

\title{Evaluación del informe\\"Generación de códigos únicos para unidades geográficas"}
\author{Sylvain Lesage\\
  \texttt{slesage@adsib.gob.bo}}
\date{\today}
\maketitle
 
\begin{abstract}
Abstract...
\end{abstract}

\section{Antecedentes}

\section{Características de la propuesta INE}

Coordenadas en la proyección cónica conforme de Lambert \cite[p.~33]{sunit07}. Ver el cuadro \ref{tab:lambert}.

\begin{table}
	\centering
	\begin{tabular}{|l|l|}
		\hline
		Falso origen & lat: 24º S, lon: 64º \\
		Meridiano central & 64º00' Oeste \\
		Latitud de origen & 24º00' Sur \\
		1ro Paralelo estándar & 11º30' Sur \\
		2do Paralelo estándar & 21º30' Sur \\
		Falso Este & 1.000.000m \\
		Falso Norte & 0m \\
		\hline	
	\end{tabular}
	\caption{Parámetros de la proyección cónica conforme de Lambert para Bolivia}
	\label{tab:lambert}
\end{table}

Cada polígono (manzano, comunidad) esta resumido a un solo punto: su centroide. Se lo calcula con Quantum GIS versión Dufour.

Cálculos - ver programa ... (gitlab)

\subsection{Definición del código}

Dados \(x\) y \(y\) las coordenadas de un punto en la proyección cónica conforme de Lambert, el código único \(z\) esta definido por:

\[z = \lfloor r \exp{\theta} \rfloor\]

donde

\[r = \sqrt{x^2+y^2} \]

y

\[\theta = \arctan{\frac{y}{x}} \]

\subseccion{Límites de Bolivia}

Para Bolivia, los valores de \(r\) y \(\theta\) se encuentran en los intervalos siguientes \cite{lesage14a}.



\section{Alternativas}

\section{Tabla comparativa entre los códigos}

\section{Conclusión y recomendaciones}

\begin{thebibliography}{99}

\bibitem{sunit07}
  Sistema Unico de Información de la Tierra,
  \emph{Normas técnicas para la administración de la información georeferenciada a nivel nacional}.
  Resolución Ministerial Nº~338 del MDRAyMA,
  2007.

\end{thebibliography}

\end{document}
